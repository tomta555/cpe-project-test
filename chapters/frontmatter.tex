\maketitle
\makesignature

\ifproject
\begin{abstractTH}
เป้าหมายหลักของการจัดการสอบปลายภาคนั้นก็เพื่อที่จะวัดความรู้ความสามารถของนักศึกษา หลังจากจบภาคการศึกษาแล้ว
หากทางมหาวิทยาลัยทำการจัดตารางสอบไม่ดีอาจจะทำให้นักศึกษาบางคนจะต้องสอบสองหรือสามวิชาติดกันในวันเดียว
หรือนักศึกษาบางคนมีวันเว้นว่างระหว่างแต่ละวิชาที่สอบมากเกินไป ซึ่งความไม่สมดุลของตารางสอบนี้อาจจะทำให้นักศึกษาเกิดความเครียดหรือลดทอนกำลังใจในการสอบของนักศึกษาได้
ระบบจัดตารางสอบปลายภาคนี้ เป็นโปรแกรมที่นำวิธีการแก้ปัญหาการระบายสีกราฟมาประยุกต์ใช้ในการจัดตารางสอบปลายภาคจากข้อมูลการลงทะเบียนของนักศึกษาหลังจากการลงทะเบียนเสร็จสิ้นแล้ว เพื่อให้ได้ตารางสอบที่มีการกระจายตัวของวิชาที่สอบอย่างมีความสมดุล
ลดจำนวนนักศึกษาที่ต้องสอบสองวิชาติดกันในหนึ่งวันให้น้อยที่สุด โดยตารางสอบของนักศึกษาคนใด ๆ ต้องไม่มีวิชาที่สอบตรงกัน ช่วยลดช่วงวันเว้นว่างของนักศึกษาในการสอบ 
และยังช่วยให้นักศึกษาสามารถเลือกลงทะเบียนวิชาเรียนที่สนใจได้อย่างอิสระโดยไม่ต้องกังวลว่าตารางสอบของวิชาที่ต้องการลงทะเบียนจะทับซ้อนกัน
\CIreply{เขียนผลลัพธ์จากการ implement เช่น เวลาที่ใช้ และผลการสำรวจความพึงพอใจเพิ่มโดยสังเขป}
\end{abstractTH}

\begin{abstract}
The main goal of scheduling final exams is to have every student take an exam for every course they registered for that semester to assess their knowledge and learning ability.
Without careful planning, the final exam schedule for a particular student might require that the student take two or three final exams consecutively or create a gap too long between two exams make an imbalance that affects the stress level of the student.
"Final exam scheduling system" is a program that applies a graph-coloring algorithm to solve the final exam scheduling problem. The system uses data from student registration to create a balanced final exam schedule. The goal of this system is to reduce the number of students who need to take exams consecutively,
to reduce the gap between each exam, and—importantly—to enable students to enroll in any course they want to without worrying about the final exam schedule. The final exam schedule for every student produced from this system must not have overlapping courses.
\CIreply{CI TODO: check after rewriting Thai abstract}
\end{abstract}

\iffalse
\begin{dedication}
This document is dedicated to all Chiang Mai University students.

Dedication page is optional.
\end{dedication}
\fi % \iffalse

\begin{acknowledgments}
โครงงานนี้ได้รับความกรุณาจาก อ.ดร.ชินวัตร อิศราดิสัยกุล อาจารย์ที่ปรึกษา ที่ได้สละเวลาให้ความช่วยเหลือทั้งให้คำแนะนำ ให้ความรู้และแนวคิดต่าง ๆ รวมถึง อ.ดร.พฤษภ์ บุญมา และ รศ.ดร.สรรพวรรธน์ กันตะบุตร ที่ให้คำปรึกษาจนทำให้โครงงานเล่มนี้เสร็จสมบูรณ์ไปได้
    
ขอขอบคุณงานบริการการศึกษาและพัฒนาคุณภาพนักศึกษา คณะต่าง ๆ ใน มหาวิทยาลัยเชียงใหม่ ที่ได้ให้ความร่วมมือเป็นอย่างดีในการให้ข้อมูลอันเป็นประโยชน์ต่อการพัฒนาโครงงานนี้

ขอขอบคุณเพื่อน ๆ ที่ช่วยให้ข้อมูลอันเป็นประโยชน์ต่อการทำโครงงาน และช่วยให้กำลังใจ รวมถึงคำแนะนำที่ดีตลอดการทำโครงงานที่ผ่านมา

ขอขอบคุณนักศึกษาทุกคนที่ช่วยให้ความร่วมมือในการตอบแบบสอบถาม ช่วยแสดงความคิดเห็นพร้อมทั้งข้อเสนอแนะเกี่ยวกับช่วงเวลาที่ต้องการจะสอบ ซึ่งเป็นข้อมูลที่สำคัญที่ใช่ในการออกแบบและพัฒนาโครงงานให้สำเร็จลุล่วง

ขอขอบคุณร้านอาหารต่างๆ ที่ให้สถานที่ในการประชุมแต่ละครั้ง พร้อมทั้งยังทำให้อิ่มท้องทำให้สามารถใช้ความคิดได้ดีมากยิ่งขึ้น

นอกจากนี้ผู้จัดทำขอขอบพระคุณ บิดา มารดา และครอบครัวที่ได้คอยให้คำแนะนำและ ให้กำลังใจ ตลอดมา รวมทั้งขอขอบพระคุณอีกหลาย ๆ ท่านที่ไม่ได้เอ่ยนามมา ณ ที่นี้ ที่ได้ให้ความช่วยเหลือจนทำให้โครงงานนี้สำเร็จลงได้ 


\acksign{2021}{4}{2}
\end{acknowledgments}%
\fi % \ifproject

\contentspage

\ifproject
\figurelistpage

\tablelistpage
\fi % \ifproject

% \abbrlist % this page is optional

% \symlist % this page is optional

% \preface % this section is optional
