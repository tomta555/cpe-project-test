\maketitle
\makesignature

\ifproject
\begin{abstractTH}
\CI{ปัจจุบันตารางสอบของนักศึกษาทุกระดับในมหาวิทยาลัยเชียงใหม่ มีการจัดตารางสอบโดยแบ่งกระบวนวิชาออกเป็นสองส่วน 
ได้แก่ กระบวนวิชาปกติคือกระบวนวิชาที่มีตอนเดียว และกระบวนวิชาพิเศษคือกระบวนวิชาที่มีมากกว่าหนึ่งตอน
โดยกระบวนวิชาปกติจะถูกนำไปจัดตารางสอบโดยอ้างอิงตามช่วงเวลาที่เรียนของแต่ละกระบวนวิชา ซึ่งถูกกำหนดเอาไว้ชัดเจน
ส่วนของกระบวนวิชาพิเศษจะนำไปจัดตารางสอบโดยเลือกเวลาจัดสอบของแต่ละกระบวนวิชานั้นที่ไม่ตรงกับเวลาจัดสอบของกระบวนวิชาปกติและอยู่ในช่วงของสัปดาห์ที่จัดการสอบ
เพื่อแก้ปัญหาให้วิชาที่มีมากกว่าหนึ่งตอนนั้นสามารถสอบในเวลาเดียวกันได้}{ก็อาจจะยังละเอียดไปอยู่ดี แค่คงแบบนี้ไว้ก่อนก็ได้}
จากการจัดตารางสอบด้วยวิธีข้างต้นนี้ทำให้ตารางสอบของนักศึกษาบางคนเกิดปัญหาขึ้น
ซึ่งทำให้นักศึกษาไม่สามารถเลือกลงทะเบียนกระบวนวิชาที่สนใจได้อย่างอิสระ เนื่องจากตารางสอบที่ซับซ้อนกัน
เวลาสอบติดกันตั้งแต่สองวิชาขึ้นไปในวันเดียว อีกทั้งบางกระบวนวิชาปกตินั้นยังมีการกำหนดวันสอบอยู่ในช่วงวันสุดท้ายของสัปดาห์ที่จัดสอบทำให้นักศึกษาบางคนว่าง
ระหว่างกลางสัปดาห์สอบเป็นเวลาหลายวัน จากปัญหาที่ได้กล่าวมานั้น ทางทีมผู้จัดทำจึงได้พัฒนาระบบจัดตารางสอบขึ้นมา 
โดยใช้ข้อมูลจากการลงทะเบียนของนักศึกษา จำนวนกระบวนวิชาทั้งหมดที่เปิดสอน และคู่กระบวนวิชาที่มีนักศึกษาลงทะเบียนทั้งสองกระบวนวิชาพร้อมกัน 
เพื่อนำมาสร้างข้อจำกัดสำหรับนำมาใช้กับโปรแกรมแก้ปัญหา(Solver ?) เพื่อหาตารางสอบที่ดีที่สุดที่เป็นไปได้จากข้อจำกัดต่าง ๆ 
โดยสามารถประเมินว่าตารางสอบเป็นตารางสอบที่ดีได้หากตารางสอบที่ได้จากโปรแกรมแก้ปัญหามีค่า penalty ? น้อย 

\textcolor{Orange2}{
ปัจจุบัน การจัดตารางสอบของนักศึกษาทุกระดับในมหาวิทยาลัยเชียงใหม่นั้นแบ่งกระบวนวิชาทั้งหมดออกเป็นสองประเภท ได้แก่ กระบวนวิชาปกติ ซึ่งจัดสอนเพียงตอนเดียว และกระบวนวิชาพิเศษ ซึ่งจัดสอนหลายตอน
\enskip
ตารางสอบของกระบวนวิชาปกตินั้นจะอ้างอิงตามช่วงเวลาเรียน
ส่วนตารางสอบของกระบวนวิชาพิเศษ จะเป็นช่วงเวลาที่ไม่ตรงกับเวลาสอบของกระบวนวิชาปกติ เพื่อให้นักศึกษาที่ลงทะเบียนวิชาเดียวกันในเวลาที่ต่างกันสามารถสอบในเวลาเดียวกันได้
\enskip
การจัดตารางสอบในลักษณะดังกล่าวทำให้นักศึกษาไม่สามารถลงทะเบียนกระบวนวิชาที่สนใจได้อย่างอิสระ เนื่องจากตารางสอบจะเกิดการทับซ้อน หรือเวลาสอบของอย่างน้อยสองวิชานั้นติดกันมากเกินไป
\enskip
นอกจากนี้ กระบวนวิชาปกติบางวิชานั้นยังกำหนดวันสอบให้อยู่ในช่วงท้ายๆ ของฤดูกาลสอบ ทำให้นักศึกษาบางคนจำเป็นต้องอยู่รอสอบโดยไม่มีการสอบวิชาอื่นๆ ในช่วงก่อนหน้าด้วย
\enskip
จากปัญหาข้างต้น คณะผู้จัดทำจึงได้พัฒนาระบบจัดตารางสอบโดยใช้ข้อมูลลงทะเบียนของนักศึกษา จำนวนกระบวนวิชาทั้งหมดที่เปิดสอน และคู่กระบวนวิชาที่มีนักศึกษาลงทะเบียนพร้อมกัน
เพื่อนำมาสร้างสมการเงื่อนไขสำหรับโปรแกรมที่ใช้หาคำตอบที่ดีที่สุด และนำมาสร้างฟังก์ชันวัตถุประสงค์สำหรับประเมินและเปรียบเทียบคุณภาพของตารางสอบที่แตกต่างกัน
}
\end{abstractTH}

\begin{abstract}
The abstract would be placed here. It usually does not exceed 350 words
long (not counting the heading), and must not take up more than one (1) page
(even if fewer than 350 words long).

Make sure your abstract sits inside the \texttt{abstract} environment.
\end{abstract}

\iffalse
\begin{dedication}
This document is dedicated to all Chiang Mai University students.

Dedication page is optional.
\end{dedication}
\fi % \iffalse

\begin{acknowledgments}
โครงงานนี้ได้รับความกรุณาจาก อ.ดร.ชินวัตร อิศราดิสัยกุล อาจารย์ที่ปรึกษา ที่ได้สละเวลาให้ความช่วยเหลือทั้งให้คำแนะนำ ให้ความรู้และแนวคิดต่าง ๆ รวมถึง อ.ดร.พฤษภ์ บุญมา และ รศ.ดร.สรรพวรรธน์ กันตะบุตร ที่ให้คำปรึกษาจนทำให้โครงงานเล่มนี้เสร็จสมบูรณ์ไปได้
    
ขอขอบคุณเพื่อนๆ ที่ช่วยให้ข้อมูลอันเป็นประโยชน์ต่อการทำโครงงาน และช่วยให้กำลังใจ รวมถึงคำแนะนำที่ดีตลอดการทำโครงงานที่ผ่านมา

นอกจากนี้ผู้จัดทำขอขอบพระคุณขอขอบพระคุณบิดา มารดาที่ได้ให้ชีวิต เลี้ยงดูสั่งสอน และส่งเสียให้กระผมได้ศึกษาเล่าเรียนจนจบหลักสูตรปริญญาตรี 
หลักสูตรวิศวกรรมศาสตรบัณฑิต ซึ่งท่านได้ให้กำลังใจ ในวันที่ท้อแท้ตลอดมา ซึ่งท่านยังเป็นแรงผลักดันให้กระผมสร้างสรรค์และมุ่งมั่นจนทำให้โครงงานนี้สำเร็จ 
รวมทั้งขอขอบพระคุณอีกหลายๆท่านที่ไม่ได้เอ่ยนามมา ณ ที่นี้ ที่ได้ให้ความช่วยเหลือตลอดมา 
หากหนังสือโครงงานเล่มนี้มีข้อผิดพลาดประการใด กระผมขอน้อมรับด้วยความยินดี

\acksign{2020}{5}{25}
\end{acknowledgments}%
\fi % \ifproject

\contentspage

\ifproject
\figurelistpage

\tablelistpage
\fi % \ifproject

% \abbrlist % this page is optional

% \symlist % this page is optional

% \preface % this section is optional
