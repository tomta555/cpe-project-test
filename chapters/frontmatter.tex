\maketitle
\makesignature

\ifproject
\begin{abstractTH}
การจัดตารางสอบสำหรับทุกรายวิชาที่เปิดสอนภายในมหาวิทยาลัยนั้น
เป็นเรื่องยากที่จะจัดการไม่ให้\CI{เกิดการชนกันของบางรายวิชาเกิดขึ้น}{ชนกัน แปลว่าอะไร} 
ทางสำนักทะเบียนจึงใช้วิธีการจัดตารางสอบโดยการใช้ตารางที่เคยจัดไว้แล้วในแต่ละปีมาปรับใช้เพิ่มเติมกับตารางสอบปีปัจจุบัน 
โดยหากมีรายวิชาเพิ่มมาในเทอมนี้ที่ไม่เคยเปิดสอนในเทอมเดียวกันของปีก่อน รายวิชานั้นจะถูกนำมาจัดช่วงเวลาสอบ\CI{ภายหลังจากที่รายวิชาที่เคยมีอยู่แล้วได้รับการจัดให้ลงตัว}{ทำงี้จริงเหรอ} อีกทั้งตารางสอบที่ใช้อยู่ในปัจจุบันถูกจัดให้แต่ละวันมีช่วงเวลาในการสอบ 3 ช่วงเวลา
จึงทำให้นักศึกษาบางคนอาจจะต้องสอบถึง 3 รายวิชาในหนึ่งวัน และเนื่องจากตารางสอบของแต่ละปีนั้นใช้งานได้อยู่แล้ว
จึงทำให้ทางสำนักทะเบียนไม่ได้เปลี่ยนวิธีการจัดตารางสอบ ซึ่งทำให้แต่ละรายวิชาที่มีการเปิดสอนมักจะถูกจัดให้สอบในช่วงเวลาเดิม ๆ
ทำให้นักศึกษาไม่สามารถลงทะเบียนเรียนบางรายวิชาพร้อมกันได้ เนื่องจากคู่รายวิชานั้นจะต้องถูกจัดให้สอบในช่วงเวลาเดียวกันตามตารางสอบของปีที่ผ่าน ๆ มา 
หรือในบางรายวิชาที่เป็นวิชาบังคับที่นักศึกษาจะต้องลงคู่กันอาจจะมีบางคู่วิชาที่ถูกจัดให้สอบต่อกันในวันเดียวกัน
การจัดตารางสอบด้วยระบบแบบนี้จึงเป็นการผลักภาระให้กับนักศึกษาต้องถูกบังคับให้เลือกลงทะเบียนเรียนเฉพาะวิชาที่ตารางสอบไม่ชนกันเท่านั้น 
\CI{}{ที่เขียนมานี้ น่าไปอยู่ใน intro 1.1 มาก}
ทางผู้จัดทำจึงได้จัดทำระบบจัดตารางสอบปลายภาค ซึ่งเป็นระบบที่ช่วยจัดการปัญหาการจัดตารางสอบของนักศึกษา
\CI{ตั้งแต่ระดับปริญญาตรี รวมทั้งระดับปริญญาโทและระดับปริญญาเอกด้วย}{ทุกระดับในมหาวิทยาลัย} ซึ่งจะช่วยให้บางคู่รายวิชาที่เปิดสอนไม่จำเป็นต้องถูกบังคับให้สอบตรงกันเสมอหากมีนักศึกษาที่ลงทะเบียนเรียน
ทั้งสองวิชานี้พร้อม ๆ กัน ทำให้นักศึกษาทุกคนสามารถเลือกเรียนวิชาที่ตนเองสนใจในแต่ละเทอมได้ โดยไม่มีข้อจำกัดด้วยช่วงเวลาสอบที่ชนกัน
โดยทางด้านของระบบจะพยายามจัดเวลาสอบให้เหมาะสมกับนักศึกษาทุกคนมากที่สุดเท่าที่จะเป็นไปได้ โดยสำหรับตารางสอบของนักศึกษาคนใด ๆ ก็ตาม 
จะพยายามไม่ให้ในหนึ่งวันมีการจัดสอบสามรายวิชาที่ลงทะเบียน และสำหรับวันที่มีการจัดสอบสองรายวิชานั้น จะพยายามไม่ให้สองรายวิชาอยู่ในช่วงเวลาที่ติดกัน
\CI{โดยหากเป็นไปได้จะจัดตารางสอบให้เหลือเพียงวันละ 2 ช่วงเวลา ได้แก่ ช่วงเช้าและช่วงเย็นเท่านั้น}{ยังละเอียดเกินไปใน abstract}
\CI{}{เรียงลำดับความคิดดังนี้
\begin{itemize}[nosep,leftmargin=*]
    \item เกริ่นถึงวิธีการจัดตารางสอบในปัจจุบัน (เน้น regular exam ที่จัดตามตารางเรียน แล้วพูดถึง special ที่พยายามแก้ปัญหาวิชาที่มีหลาย sections)
    \item พูดถึงปัญหาที่เกิดขึ้น (สอบชนกัน สอบหลายวิชาในวันเดียวกัน สอบติดกันหลายวิชา ว่างสอบนานหลายวัน)
    \item จะ focus เฉพาะ มช หรือไม่ หรือใช้ได้กับที่อื่นด้วย?
    \item แนะนำระบบ ที่จะช่วยลดปัญหาดังกล่าว
    \item ระบบใช้ข้อมูลอะไรบ้างที่จะช่วยทำให้การสอบ balanced มากขึ้น
    \item ใช้วิธีอะไรในการจัดตารางสอบ (ปรับได้ทีหลัง)
    \item พูดถึงการทดสอบและประเมินระบบ? (ปรับได้ทีหลัง)
\end{itemize}
}
% \\จัดการระบบโดยใช้อัลกลอริทึมที่มีความเป็นไปได้ รวมทั้งการเก็บความต้องการของนักศึกษา
\end{abstractTH}

\begin{abstract}
The abstract would be placed here. It usually does not exceed 350 words
long (not counting the heading), and must not take up more than one (1) page
(even if fewer than 350 words long).

Make sure your abstract sits inside the \texttt{abstract} environment.
\end{abstract}

\iffalse
\begin{dedication}
This document is dedicated to all Chiang Mai University students.

Dedication page is optional.
\end{dedication}
\fi % \iffalse

\begin{acknowledgments}
Your acknowledgments go here. Make sure it sits inside the
\texttt{acknowledgment} environment.

\acksign{2020}{5}{25}
\end{acknowledgments}%
\fi % \ifproject

\contentspage

\ifproject
\figurelistpage

\tablelistpage
\fi % \ifproject

% \abbrlist % this page is optional

% \symlist % this page is optional

% \preface % this section is optional
