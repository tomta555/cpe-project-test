\chapter{\ifcpe บทนำ\else Introduction\fi}

\section{\ifcpe ที่มาของโครงงาน\else Project rationale\fi}


การสอบปลายภาคหรือการทดสอบความรู้ในช่วงท้ายของการเรียนเป็นส่วนหนึ่งในการวัดระดับความรู้ที่นักศึกษาได้เรียนรู้ไปในวิชานั้น ๆ
ซึ่งสามารถบ่งบอกถึงความเข้าใจ เอาใจใส่ของนักศึกษาคนนั้น ๆ ซึ่งปัจจัยที่มีผลต่อการสอบนั้นมีอยู่หลายปัจจัย
ตัวอย่างเช่น ความตั้งใจ ความพยายาม ความเอาใจใส่กับรายวิชาของนักศึกษา
แต่ก็ยังมีปัจจัยอีกหนึ่งปัจจัยที่มีผลต่อการสอบปลายภาคด้วย คือตารางสอบปลายภาคของนักศึกษา
โดยตารางสอบนั้นอาจจะส่งผลกระทบต่อนักศึกษาหากทางมหาวิทยาลัยมีการจัดตารางสอบไม่ดีพอ
ซึ่งในที่นี้จะกล่าวถึงการจัดตารางสอบของมหาวิทยาลัยเชียงใหม่ ซึ่งในแต่ละภาคการศึกษานั้นมีนักศึกษาที่เรียนมากกว่า 35,000 คน 
มีวิชาที่เปิดให้ลงทะเบียนมากกว่า 3000 วิชา มีคู่วิชาที่มีนักศึกษาลงทะเบียนพร้อมกันอย่างน้อย 1 คน มากกว่า 30000 คู่วิชา 
โดยตารางสอบนั้นจะเป็นตารางที่ถูกจัดไว้ก่อนการลงทะเบียนเรียนของนักศึกษา 
และการลงทะเบียนเรียนนักศึกษาจะต้องเลือกลงทะเบียนไม่ให้มีวิชาที่สอบตรงกัน
ไม่อย่างนั้นหากนักศึกษาไม่ได้พิจารณาเวลาสอบให้ดีก่อนการลงทะเบียน นักศึกษาจะต้องถอนวิชาใดวิชาหนึ่งที่เวลาสอบตรงกันออก
หรือจะได้รับผลการประเมินขั้น ``F'' ในวิชาที่ไม่สามารถเข้าสอบปลายภาคได้


ทางมหาวิทยาลัยเชียงใหม่คาดหวังว่านักศึกษาจะเลือกลงทะเบียนเรียนวิชาต่าง ๆ
โดยเลือกลงวิชาที่เวลาเรียนไม่ตรงกัน ดังนั้นทางมหาวิทยาลัยจึงจัดตารางสอบของนักศึกษาโดยจัดตารางให้กับวิชาที่มีตอนเดียวก่อน
(เรียกว่า Regular Exam) โดยจะอ้างอิงตามช่วงเวลาที่เรียน วิชาที่เรียนในวันและเวลาเดียวกันจะถูกจัดให้สอบในช่วงเวลาเดียวกัน
ส่วนวิชาที่มีหลายตอน (เรียกว่า Special Exam)\CIreply{ไม่ได้เรียก ``วิชา'' พวกนี้ว่า special exam} จะถูกจัดให้สอบในช่วงเวลาที่ไม่ตรงกับเวลาสอบของวิชาที่มีตอนเดียว
เพื่อให้นักศึกษาที่ลงทะเบียนวิชาเดียวกันในเวลาที่ต่างกัน  (ลงทะเบียนวิชาเดียวกันแต่คนละตอน) สามารถสอบในเวลาเดียวกันได้
เพื่อเป็นการป้องกันข้อสอบรั่วไหล ซึ่งการจัดตารางสอบในลักษณะดังกล่าวนั้นทําให้นักศึกษาไม่สามารถลงทะเบียนวิชาที่สนใจได้อย่างอิสระ
เนื่องจากตารางสอบตรงกันหรือเวลาสอบของสองวิชาที่ต้องการลงทะเบียนนั้นติดกันมากเกินไป 
นอกจากนี้ การจัดตารางสอบด้วยวิธีนี้ ทำให้บางวิชานั้นสามารถกําหนดวันสอบให้อยู่ในช่วงท้าย ๆ ของฤดูกาลสอบได้ 
ทําให้นักศึกษาบางคนมีวันเว้นว่างระหว่างการสอบหลายวันเพราะจําเป็นต้องอยู่รอทำการสอบในวิชาท้าย ๆ
จากในงานวิจัยของ Ender Ozcan (2005)\CIreply{cite properly} ได้เคยมีการสร้างโปรแกรมสำหรับจัดตารางสอบมาก่อนโดยมีการจัดตารางสอบแบ่งเป็นตารางสอบของวิชาภายในคณะและวิชาภายในภาควิชา 
ซึ่งไม่สามารถใช้ได้กับวิชาเรียนของทางมหาวิทยาลัยเชียงใหม่ 
เนื่องจากมีวิชาด้านการศึกษาทั่วไปที่มีนักศึกษาจากต่างคณะสามารถลงทะเบียนเรียนร่วมกันได้


จากปัญหาข้างต้น ทางผู้จัดทำคาดว่าการพัฒนาระบบจัดตารางสอบโดยใช้ข้อมูลลงทะเบียนของนักศึกษา 
หลังจากการลงทะเบียนเสร็จสิ้นแล้ว จะช่วยให้นักศึกษามีอิสระในการเลือกลงวิชาเรียนมากขึ้น 
ช่วยลดจำนวนนักศึกษาที่ต้องสอบสองวิชาติดกันในหนึ่งวันให้น้อยลง และลดวันเว้นว่างระหว่างช่วงเวลาสอบ 
ซึ่งจะช่วยลดความเครียดส่วนหนึ่งในช่วงสอบของนักศึกษาได้

\newpage
\section{\ifcpe วัตถุประสงค์ของโครงงาน\else Objectives\fi}
\begin{enumerate}
    \item เพื่อพัฒนาโปรแกรมสำหรับจัดตารางสอบปลายภาคในมหาวิทยาลัย จากข้อมูลการลงทะเบียนของนักศึกษา โดยมีตารางสอบที่ได้มีคุณสมบัติ ดังนี้
    \begin{itemize}
        \item ต้องไม่มีนักศึกษาคนใด ๆ มีวิชาที่สอบสองวิชาในช่วงเวลาเดียวกัน
        \item จำนวนนักศึกษาที่มีสอบสองวิชาติดกันในวันเดียวลดลงจากตารางสอบดั้งเดิม
        \item จำนวนนักศึกษาที่มีช่วงวันที่เว้นจากการสอบวิชาก่อนหน้า นานเกินไป\CIreply{เมื่อไรคือนานเกินไป} ลดลงจากตารางสอบดั้งเดิม
        \item ในแต่ละช่วงเวลาที่จัดสอบ จำนวนนักศึกษาที่สอบจะต้องไม่เกินจำนวนที่นั่งสอบที่ทางมหาวิทยาลัยสามารถจัดให้ได้
    \end{itemize}
    \item เพื่อแก้ไขปัญหานักศึกษาไม่สามารถลงทะเบียนในรายวิชาที่ต้องการได้อย่างอิสระ เนื่องจากตารางสอบของรายวิชาที่ต้องการลงทะเบียนอยู่ในช่วงเวลาเดียวกัน
    \item เพื่อลดจำนวนนักศึกษาที่ต้องสอบสองวิชาติดกันในหนึ่งวันให้น้อยที่สุด
    \item เพื่อลดจำนวนนักศึกษาที่มีวันเว้นว่างระหว่างสอบหลายวัน\CIreply{หลายวันคือกี่วัน (ซ้ำกับข้างต้นไหม)}ให้น้อยที่สุด
    \item เพื่อช่วยลดภาระในการจัดตารางสอบของคณาจารย์\CIreply{ลดอย่างไร}
\end{enumerate}

\section{\ifcpe ขอบเขตของโครงงาน\else Project scope\fi}
เป้าหมายของโครงงานนี้ ต้องการพัฒนาโปรแกรมสำหรับจัดตารางสอบปลายภาคในระดับมหาวิทยาลัย
โดยโครงงานนี้จะพิจารณาความต้องการและข้อจำกัดต่าง ๆ ที่ใช้ในการจัดตารางสอบของมหาวิทยาลัยเชียงใหม่
และจะพิจารณาเฉพาะข้อจำกัดทางด้านเวลาเท่านั้น โดยมีขอบเขตของโครงงานดังนี้ 
\subsection{\ifcpe ขอบเขตด้านฮาร์ดแวร์\else Hardware scope\fi}
\begin{enumerate}
    \item คอมพิวเตอร์ที่มีความสามารถในการประมวลผลสูง\CIreply{สูงขนาดไหน} เพื่อลดเวลาในการประมวลผลตารางสอบ
\end{enumerate}
\subsection{\ifcpe ขอบเขตด้านซอฟต์แวร์\else Software scope\fi}
\CIreply{real-time? เวลาที่ใช้?}
\begin{enumerate}
    \item ระบบต้องใช้เวลาอย่างน้อย 8 ชั่วโมง ในการประมวลผลตารางสอบด้วย Optimizer\CIreply{realistic?}
\end{enumerate}
\subsection{ข้อจำกัดที่นำมาพิจารณาในการจัดตารางสอบ}
\begin{enumerate}
    \item ระยะเวลาที่ใช้ในการจัดสอบปลายภาค คือ 2 สัปดาห์
    \item จำนวนช่วงเวลาที่จัดสอบในแต่ละวัน คือ 3 ช่วงเวลา ได้แก่ 8:00--11:00, 12:00--15:00 
    และ 15:30--18:30
    \item จำนวนที่นั่งสอบที่มหาวิทยาลัยสามารถจัดให้ได้
    \item จำนวนที่นั่งสอบในแต่ละห้องสอบของแต่ละคณะ
    \item วิชาเรียนที่มีการจัดสอบปลายภาค
    \item ข้อมูลจากการลงทะเบียนล่วงหน้าและลงทะเบียนรอบปกติ
    \item คู่วิชาเรียนที่มีนักศึกษาลงทะเบียนร่วมกันทั้งสองวิชา โดยคู้่วิชาที่มีจำนวนนักศึกษาลงทะเบียนมาก จะมีลำดับความสำคัญสูงกว่า
\end{enumerate}

\subsection{ข้อจำกัดที่ไม่นำมาพิจารณาในการจัดตารางสอบ}
\begin{enumerate}
    \item การจัดอาจารย์ผู้คุมสอบให้กับแต่ละรายวิชา
    \item เวลาที่จัดการเรียนการสอนของแต่ละรายวิชา
    \item การจัดตารางสอบแยกตามรายวิชาของแต่ละคณะ หรือแต่ละภาควิชา
    \item ข้อมูลจากการลงทะเบียนเพิ่มเติมหลังกำหนด
\end{enumerate}

\section{\ifcpe ประโยชน์ที่ได้รับ\else Expected outcomes\fi}
\begin{enumerate}
    \item นักศึกษาสามารถเลือกลงทะเบียนวิชาที่ต้องการได้อย่างอิสระ โดยไม่ต้องกังวลเรื่องตารางสอบจะถูกจัดให้สอบในช่วงเวลาเดียวกัน
    \item ช่วยให้ตารางสอบของนักศึกษาคนใด ๆ มีจำนวนรายวิชาที่ต้องสอบในแต่ละวันน้อยลง
    \item ช่วยลดวันเว้นว่างระหว่างช่วงเวลาสอบของนักศึกษาให้น้อยลง
    \item ช่วยลดภาระในการจัดตารางสอบของคณาจารย์ 
\end{enumerate}
\CIreply{น่าจะต้องปรับเป็นร้อยแก้วทีหลัง}

\section{\ifcpe เทคโนโลยีและเครื่องมือที่ใช้\else Technology and tools\fi}

\subsection{\ifcpe เทคโนโลยีด้านฮาร์ดแวร์\else Hardware technology\fi}
\begin{enumerate}
    \item Cloud Computing ?
\end{enumerate}
\subsection{\ifcpe เทคโนโลยีด้านซอฟต์แวร์\else Software technology\fi}
\begin{enumerate}
    \item Google OR-Tools
    \item Gurobi Optimizer
\end{enumerate}

\section{\ifcpe แผนการดำเนินงาน\else Project plan\fi}

\begin{plan}{7}{2020}{3}{2021}
    \planitem{7}{2020}{8}{2020}{Literature Review}
    \planitem{8}{2020}{9}{2020}{เขียนอัลกอริทึม เวอร์ชัน 1}
    \planitem{8}{2020}{10}{2020}{เก็บข้อมูลครั้งที่ 1}
    \planitem{9}{2020}{10}{2020}{เขียนอัลกอริทึม เวอร์ชัน 2}
    \planitem{10}{2020}{10}{2020}{เขียนรายงาน}
    \planitem{11}{2020}{12}{2020}{เก็บข้อมูลครั้งที่ 2}
    \planitem{11}{2020}{1}{2021}{เขียนอัลกอริทึม เวอร์ชัน 3}
    \planitem{1}{2021}{2}{2021}{ทดสอบและประเมินผลอัลกอริทึม}
    \planitem{1}{2021}{2}{2021}{ทดสอบความพึงพอใจของนักศึกษา}
    \planitem{2}{2021}{3}{2021}{เขียนรายงาน}
\end{plan}

\section{\ifcpe บทบาทและความรับผิดชอบ\else Roles and responsibilities\fi}
อธิบายว่าในการทำงาน นศ. มีการกำหนดบทบาทและแบ่งหน้าที่งานอย่างไรในการทำงาน จำเป็นต้องใช้ความรู้ใดในการทำงานบ้าง

\section{\ifcpe%
ผลกระทบด้านสังคม สุขภาพ ความปลอดภัย กฎหมาย และวัฒนธรรม
\else%
Impacts of this project on society, health, safety, legal, and cultural issues
\fi}

ผลกระทบทางด้านสังคม: ระบบที่เป็นผลลัพธ์ของโครงงานนี้จะสามารถใช้เป็นทางเลือกในการจัดการตารางสอบในระดับมหาวิทยาลัยได้
ซึ่งตารางสอบที่ได้จากโปรแกรมนี้จะช่วยให้นักศึกษาสามารถเลือกเรียนวิชาที่ตนเองต้องการได้อย่างอิสระ โดยไม่มีข้อจำกัดด้านตารางสอบ
เหมือนอย่างระบบเดิมที่ใช้อยู่ในปัจจุบัน