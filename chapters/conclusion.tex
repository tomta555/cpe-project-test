\chapter{\ifcpe บทสรุปและข้อเสนอแนะ\else Conclusions and Discussion\fi}

\section{\ifcpe สรุปผล\else Conclusions\fi}

/สรุปเรื่องผลจาก penalty และ ผลจากสำรวจ นศ

ปัจจุบันโครงงานนี้ได้รับการพัฒนาเสร็จสิ้นจนใช้งานได้แล้ว แต่ด้วยความที่ระบบนั้นมีการใช้ข้อมูลการลงทะเบียนของนักศึกษาจำนวนมาก 
และข้อมูลรายวิชาที่มีการจัดสอบซึ่งได้รวบรวมมาจากตารางสอบในระยะเวลา 3 ปี ย้อนหลัง ของแต่ละคณะ 
มาใช้เพื่อคัดกรองรายวิชาที่ต้องนำมาจัดตารางสอบและใช้เป็นข้อมูลนำเข้าของโปรแกรม ข้อมูลส่วนนี้จึงต้องมีความถูกต้องสูงมากเพื่อให้ตารางสอบที่ออกมาสามารถนำไปใช้งานได้จริง
ข้อมูลเหล่านี้จึงค่อนข้างเป็นอุปสรรค หากในอนาคตมีการเพิ่มรายวิชาใหม่ ๆ หรือมีการเปลี่ยนแปลงรายวิชาต่าง ๆ จากการปรับปรุงหลักสูตร ก็จะต้องได้รับข้อมูลในส่วนนั้นเพิ่มเติมด้วย เพื่อให้โปรแกรมทำงานได้ถูกต้องตามที่ต้องการ

\section{\ifcpe ปัญหาที่พบและแนวทางการแก้ไข\else Challenges\fi}

ในการทำโครงงานนี้ พบว่าเกิดปัญหาหลักๆ ดังนี้
\subsection{ความซับซ้อนและปริมาณของข้อมูล}
เนื่องจากการจะจัดตารางสอบขึ้นมาได้เราจะต้องทราบว่ารายวิชาใดบ้างที่มีการจัดสอบในแต่ละเทอม 
แต่บางรายวิชาในข้อมูลของสำนักทะเบียนมีการระบุว่ามีสอบแต่จริง ๆ แล้ววิชานั้นผู้สอนไม่ได้จัดสอบหรือมีการจัดสอบนอกตาราง
จึงต้องมีการส่งคำร้องเพื่อขอข้อมูลเหล่านั้นจากแต่ละคณะภายในมหาวิทยาลัย ซึ่งแต่ละคณะก็ใช้เวลาดำเนินการรวบรวมและส่งข้อมูลกลับมาไม่เท่ากัน
อีกทั้งเนื่องจากในแต่ละคณะมีการจัดเก็บข้อมูลไว้ในรูปแบบที่แตกต่างกัน และในแต่ละเทอมมหาวิทยาลัยเชียงใหม่ 
มีนักศึกษาลงทะเบียนมากกว่า 30,000 คน และมีรายวิชามากกว่า 3,500 วิชาที่ต้องตรวจสอบข้อมูลให้ถูกต้อง
ทำให้การประมวลผลและจัดการทำความสะอาดข้อมูลที่ไม่จำเป็นก่อนที่จะนำข้อมูลมาใช้งานได้ มีความล่าช้าขึ้นไปอีก 
แนวทางการแก้ไขปัญหานี้สามารถทำได้หากสำนักทะเบียนมีการจัดเก็บและรวบรวมข้อมูลรายละเอียดแยกย่อยของแต่ละวิชาจากแต่ละคณะรวมเข้าไว้ด้วยกันอย่างมีระเบียบ
และจัดทำ API สำหรับให้เรียกใช้งานข้อมูลเหล่านั้นได้
\subsection{ข้อมูลสูญหายและข้อมูลที่ได้รับมาไม่ครบถ้วน}
การส่งคำร้องเพื่อขอข้อมูลจากแต่ละคณะภายในมหาวิทยาลัย ทำให้ได้รับข้อมูลตารางสอบย้อนหลังของแต่ละคณะมา แต่ในบางคณะไม่สามารถหาข้อมูลให้ได้ครบทุกเทอมตามที่ต้องการ และคณะส่วนใหญ่ให้ข้อมูลเฉพาะวิชาที่มีสอบในระดับปริญญาตรีเท่านั้น
ทำให้การจัดตารางสอบให้กับวิชาตั้งแต่ระดับปริญญาโทขึ้นไปนั้นไม่สามารถจัดให้ได้อย่างสมบูรณ์ครบถ้วนทุกรายวิชา


\section{\ifcpe%
ข้อเสนอแนะและแนวทางการพัฒนาต่อ
\else%
Suggestions and further improvements
\fi
}

ข้อเสนอแนะเพื่อพัฒนาโครงงานนี้ต่อไป มีดังนี้
ในโครงงานนี้มีการมุ่งเน้นพัฒนาโปรแกรมเพื่อใช้ในการจัดตารางสอบให้กับแต่ละรายวิชา กล่าวคือ ในโครงงานนี้จะทำการจัดเพียงช่วงเวลาสอบ (slot) ให้กับแต่ละรายวิชาที่อยู่ในหลักสูตรปริญญาตรีเป็นหลัก ซึ่งหากมีข้อมูลของรายวิชาอื่น ๆ ที่ต้องจัดสอบเพิ่มเติมก็สามารถนำข้อมูลมาใช้เพื่อจัดสอบให้กับรายวิชาเหล่านั้นได้ไม่ยาก
สำหรับแนวทางในการพัฒนาและประยุกต์ใช้ร่วมกับงานอื่น ๆ นั้น หากมีข้อมูลสถานที่ห้องสอบและความจุนักศึกษาที่แต่ละห้องสอบสามารถรับได้อย่างละเอียดและถูกต้องครบถ้วนก็จะสามารถนำโปรแกรมไปพัฒนาต่อเพื่อขยายความสามารถของโปรแกรมให้สามารถจัดห้องสอบให้กับแต่ละรายวิชาได้ด้วย

