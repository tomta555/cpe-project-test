\chapter{\ifcpe บทสรุปและข้อเสนอแนะ\else Conclusions and Discussion\fi}

\section{\ifcpe สรุปผล\else Conclusions\fi}

จากการทำโครงงานนี้ทำให้ได้รับรู้ปัญหาเกี่ยวกับตารางสอบปลายภาคของนักศึกษาภายในมหาวิทยาลัย
ทั้งปัญหาการสอบสองวิชาในเวลาเดียวกันที่เกิดจากระบบลงทะเบียนไม่ได้ป้องกันไม่ให้นักศึกษาลงทะเบียนวิชาเหล่านั้น
ปัญหาการจัดเก็บข้อมูลของสำนักทะเบียนที่มีการแก้ไขข้อมูลด้วยมือโดยไม่มีการตรวจสอบความถูกของรูปแบบของข้อมูลก่อนการแก้ไขหรือเพิ่มข้อมูลใหม่
\CIreply{ไม่เคยกล่าวถึงจุดนี้}
\TSNAreply{เพิ่มในบทที่ 3 แล้วครับ}
และปัญหานักศึกษาไม่สามารถลงทะเบียนเรียนวิชาเลือกที่อยากเรียนได้เนื่องจากตารางสอบของบางวิชาที่ถูกจัดไว้ก่อนล่วงหน้าแล้วนั้นสอบเวลาเดียวกันกับวิชาบังคับที่นักศึกษาต้องลงทะเบียน
ถึงแม้ในโครงงานนี้จะไม่สามารถแก้ไขปัญหาด้านการจัดเก็บข้อมูลได้ แต่จากผลการทดลอง จะเห็นว่าตารางสอบที่จัดโดยโปรแกรมนั้นดีกว่าตารางสอบที่กำหนดโดยสำนักทะเบียนและประมวลผล มหาวิทยาลัยเชียงใหม่
ทั้งเวลาที่ใช้ในการจัดตารางสอบของโปรแกรมที่รวดเร็ว สามารถจัดตารางสอบให้เสร็จได้ใน 1--2 นาที ทั้งการที่สามารถจัดสอบให้ไม่มีนักศึกษาสอบสองวิชาในเวลาเดียวกันได้
และจำนวนนักศึกษาที่สอบหลายวิชาติดต่อกันยังมีน้อยกว่าด้วย อีกทั้งจากผลการสำรวจความพึงพอใจของนักศึกษาต่อตารางสอบทั้งสองแบบ ยังพบว่านักศึกษามากกว่า 61\%
ชอบตารางสอบที่จัดโดยโปรแกรมมากกว่าตารางสอบจากสำนักทะเบียนและประมวลผล มหาวิทยาลัยเชียงใหม่ เป็นการยืนยันว่าผลของการออกแบบ penalty model
หรือตัวชี้วัดที่โปรแกรมใช้คำนวณค่าความเหมาะสมของตารางสอบนั้นสอดคล้องกับความต้องการของนักศึกษาที่เป็นผู้ใช้งานจริง ๆ ผลลัพธ์ที่ได้จากโครงงานนี้จึงเป็นหลักฐานยืนยันที่เพียงพอต่อการที่สำนักทะเบียนและประมวลผล มหาวิทยาลัยเชียงใหม่
จะพิจารณาปรับเปลี่ยนวิธีการที่ใช้ในการจัดตารางสอบปลายภาคในปัจจุบันโดยเร็ว เพื่อประโยชน์ของนักศึกษาทุก ๆ คน

\section{\ifcpe ปัญหาที่พบและแนวทางการแก้ไข\else Challenges\fi}

ในการทำโครงงานนี้ พบว่าเกิดปัญหาหลักๆ ดังนี้
\subsection{ความถูกต้องของข้อมูล}
เนื่องจากระบบจัดตารางสอบปลายภาคนี้มีการใช้ข้อมูลการลงทะเบียนของนักศึกษาจำนวนมาก และใช้ข้อมูลรายวิชาที่มีการจัดสอบของแต่ละคณะ ซึ่งได้รวบรวมมาจากตารางสอบในระยะเวลา 3 ปีย้อนหลัง 
มาใช้เพื่อคัดกรองรายวิชาที่ต้องนำมาจัดตารางสอบ และใช้เป็นข้อมูลนำเข้าของโปรแกรม ข้อมูลส่วนนี้จึงต้องมีความถูกต้องสูงมากเพื่อให้ตารางสอบที่ออกมาสามารถนำไปใช้งานได้จริง
หากในอนาคตมีการเพิ่มรายวิชาใหม่ ๆ หรือมีการเปลี่ยนแปลงรายวิชาต่าง ๆ จากการปรับปรุงหลักสูตร ก็จะต้องได้รับข้อมูลในส่วนนั้นเพิ่มเติมด้วย เพื่อให้โปรแกรมทำงานได้ถูกต้องตามที่ต้องการ
และไม่เกิดเหตุการณ์ที่ทำให้ไม่มีตารางสอบสำหรับวิชาที่เพิ่งเปิดใหม่ ทางมหาวิทยาลัยจึงควรมีการเก็บข้อมูลเหล่านี้อย่างเป็นระบบและมีแอปพลิเคชันที่ออกแบบไว้เพื่อให้บุคลากรที่เกี่ยวข้องกับการทำงานด้านนี้ใช้งานสำหรับอัพเดตข้อมูลเหล่านี้ได้ง่ายขึ้น

\subsection{ความซับซ้อนและปริมาณของข้อมูล}
เนื่องจากในการจัดตารางสอบ เราจะต้องทราบว่ารายวิชาใดบ้างที่มีการจัดสอบในแต่ละภาคการศึกษา 
แต่บางรายวิชาในข้อมูลของสำนักทะเบียนฯ ระบุว่ามีการสอบ ทั้งที่จริงแล้ว วิชานั้นผู้สอนไม่ได้จัดสอบหรือมีการจัดสอบนอกตาราง
จึงต้องมีการส่งคำร้องเพื่อขอข้อมูลเหล่านั้นจากแต่ละคณะภายในมหาวิทยาลัย ซึ่งแต่ละคณะก็ใช้เวลาดำเนินการรวบรวมและส่งข้อมูลกลับมาไม่เท่ากัน
อีกทั้งเนื่องจากในแต่ละคณะมีการจัดเก็บข้อมูลไว้ในรูปแบบที่แตกต่างกัน และในแต่ละภาคการศึกษามหาวิทยาลัยเชียงใหม่ 
มีนักศึกษาลงทะเบียนมากกว่า 30000 คน และมีรายวิชามากกว่า 3500 วิชาที่ต้องตรวจสอบข้อมูลให้ถูกต้อง
ทำให้การประมวลผลและจัดการทำความสะอาดข้อมูลที่ไม่จำเป็นก่อนที่จะนำข้อมูลมาใช้งานได้ มีความล่าช้าขึ้นไปอีก 
แนวทางการแก้ไขปัญหานี้สามารถทำได้หากสำนักทะเบียนมีการจัดเก็บและรวบรวมข้อมูลรายละเอียดแยกย่อยของแต่ละวิชาจากแต่ละคณะรวมเข้าไว้ด้วยกันอย่างมีระเบียบ
และจัดทำ API สำหรับให้เรียกใช้งานข้อมูลเหล่านั้นได้


\section{\ifcpe%
ข้อเสนอแนะและแนวทางการพัฒนาต่อ
\else%
Suggestions and further improvements
\fi
}

ในโครงงานนี้มีการมุ่งเน้นพัฒนาโปรแกรมเพื่อใช้ในการจัดตารางสอบให้กับแต่ละรายวิชา กล่าวคือ ในโครงงานนี้จะทำการจัดเพียงช่วงเวลาสอบ (slot)
ให้กับแต่ละรายวิชาที่อยู่ในหลักสูตรปริญญาตรีเป็นหลัก ซึ่งหากมีข้อมูลของรายวิชาอื่น ๆ ที่ต้องจัดสอบเพิ่มเติมก็สามารถนำข้อมูลมาใช้เพื่อจัดสอบให้กับรายวิชาเหล่านั้นได้ไม่ยาก
สำหรับแนวทางในการพัฒนาและประยุกต์ใช้ร่วมกับงานอื่น ๆ นั้น หากมีข้อมูลสถานที่ห้องสอบและความจุนักศึกษาที่แต่ละห้องสอบสามารถรับได้อย่างละเอียดและถูกต้องครบถ้วน
ก็จะสามารถนำโปรแกรมไปพัฒนาต่อเพื่อขยายความสามารถของโปรแกรมให้สามารถจัดห้องสอบให้กับแต่ละรายวิชาได้ด้วย ซึ่งเป็นสิ่งที่บุลคลากรที่ทำงานในด้านนี้ต้องการ จากผลการสอบถามและเก็บข้อมูลจากแต่ละคณะ