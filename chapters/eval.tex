\chapter{\ifproject%
\ifcpe การทดลองและผลลัพธ์\else Experimentation and Results\fi
\else%
\ifcpe การประเมินระบบ\else System Evaluation\fi
\fi}

การประเมินระบบสำหรับโครงงานนี้จะเน้นไปที่การคำนวณค่าความเหมาะสมของตารางสอบที่ได้จากระบบ เพื่อเปรียบเทียบกับตารางสอบที่กำหนดโดยสำนักทะเบียนและประมวลผล 
\enskip การคำนวณค่าความเหมาะสมของตารางสอบที่ได้นั้นจะคำถึงนึงคุณสมบัติของตารางสอบตามวัตถุประสงค์ของโครงงาน ดังที่ได้ระบุไว้ในตอนที่~\ref{sec:Objectives} 
โดยสามารถประเมินความเหมาะสมได้ 2 วิธี กล่าวคือ การนับจำนวนครั้งที่ตารางสอบสำหรับนักศึกษาคนใดๆ ไม่เป็นไปตามที่พึงประสงค์ และการสอบถามความพึงพอใจของนักศึกษาที่มีต่อตารางสอบที่ได้จากระบบ เพื่อช่วยในการยืนยันว่าตัวชี้วัตที่โปรแกรมใช้คำนวณค่าความเหมาะสมของตารางสอบนั้นสอดคล้องกับความต้องการของผู้ใช้งานอย่างแท้จริง

\section{การประเมินระบบด้วย penalty}
การประเมินผลระบบจัดตารางสอบที่จะพัฒนาขึ้นมานั้นจะพิจารณาปัจจัยด้านความสมดุลและความเหมาะสมของตารางสอบที่ได้ โดยสามารถกำหนดค่าอันไม่พึงประสงค์ (penalty) ของตารางสอบแต่ละแบบที่เป็นผลลัพธ์จากระบบ
\enskip ตารางสอบที่มีความเหมาะสมมาก ควรจะมี penalty น้อย
\enskip นอกจากจะมีการคำนวณ penalty เพื่อเปรียบเทียบตารางสอบแบบต่างๆ ที่ได้จากระบบแล้ว การคำนวณในลักษณะเดียวกันนี้จะใช้กับตารางสอบดั้งเดิมดังที่ได้กำหนดโดยสำนักทะเบียนและประมวลผลอีกด้วย เพื่อยืนยืนว่าตารางสอบที่ได้จากระบบนั้นมีคุณภาพดีกว่าตารางสอบที่มีอยู่เดิม

\subsection{การคำนวณ penalty}
Penalty ของตารางสอบแต่ละแบบที่ได้จากระบบนั้น คำนวณได้จาก penalty ของตารางสอบสำหรับนักศึกษารายบุคคล ว่ามีความเหมาะสมกับนักศึกษารายนั้นๆ มากน้อยเพียงใด \enskip 
การคิดค่า penalty ของนักศึกษาแต่ละคนนั้น จะใช้ตัวชี้วัดทั้งสิ้น 5 แบบ ซึ่งมีน้ำหนักแตกต่างกันไปตามความไม่พึงประสงค์ที่สรุปได้จากผลสำรวจ เรียงตามน้ำหนักจากมากไปน้อยได้ดังนี้
\begin{enumerate}
    \item จำนวนวันที่นักศึกษาถูกกำหนดให้สอบเวลา 8.00--11.00 และ 12.00--15.00 ในวันนั้นๆ
    \item จำนวนครั้งที่มีวันเวลาสอบ 12.00-15.00 และ 15.30-18.30 ในวันเดียวกัน\CIreply{แก้}
    \item จำนวนครั้งที่มีวันเวลาสอบ 8.00-11.00 และ 15.30-18.30 ในวันเดียวกัน\CIreply{แก้}
    \item จำนวนครั้งที่มีวันเวลาสอบ 15.30-18.30 และวิชาถัดไปสอบเวลา 08.00-11.00 ในวันรุ่งขึ้น\CIreply{แก้}
    \item จำนวนครั้งที่วันเว้นว่างระหว่างการสอบสองวิชาที่ติดกันมีมากกว่า 3 วัน
    \item \CIreply{seating capacity}
\end{enumerate}

\section{การประเมินระบบโดยสอบถามความพึงพอใจของนักศึกษา}
ตัวชี้วัดที่ใช้ในการคำนวณ penalty ของตารางสอบที่ได้จากระบบนั้น อาจจะไม่ครอบคลุมความพึงประสงค์ทุกรูปแบบจากนักศึกษา หรือน้ำหนักของตัวชี้วัดที่ใช้ในการคำนวณ penalty นั้นอาจจะเป็นไปได้หลายรูปแบบ
\enskip การสอบถามความพึงพอใจของนักศึกษานั้นจะช่วยในการปรับปรุงอัลกอริทึม และเลือกใช้น้ำหนักของตัวชี้วัดที่เหมาะสมที่สุดในการจัดตารางสอบ
\CIreply{จะถามอะไรบ้าง เปรียบเทียบตารางหลายแบบจากหลายน้ำหนัก และเปรียบเทียบตารางของเรากับ สทป}
