\chapter{\ifproject%
\ifcpe การทดลองและผลลัพธ์\else Experimentation and Results\fi
\else%
\ifcpe การประเมินระบบ\else System Evaluation\fi
\fi}

การประเมินระบบสำหรับโครงงานนี้จะเน้นไปที่การคำนวณค่าความเหมาะสมของตารางสอบที่ได้จากระบบ เพื่อเปรียบเทียบกับตารางสอบที่กำหนดโดยสำนักทะเบียนและประมวลผล 
\enskip การคำนวณค่าความเหมาะสมของตารางสอบที่ได้นั้นจะคำถึงนึงคุณสมบัติของตารางสอบตามวัตถุประสงค์ของโครงงาน ดังที่ได้ระบุไว้ในตอนที่~\ref{sec:Objectives} 
โดยสามารถประเมินความเหมาะสมได้ 2 วิธี กล่าวคือ การนับจำนวนครั้งที่ตารางสอบสำหรับนักศึกษาคนใดๆ ไม่เป็นไปตามที่พึงประสงค์ และการสอบถามความพึงพอใจของนักศึกษาที่มีต่อตารางสอบที่ได้จากระบบ เพื่อช่วยในการยืนยันว่าตัวชี้วัดที่โปรแกรมใช้คำนวณค่าความเหมาะสมของตารางสอบนั้นสอดคล้องกับความต้องการของผู้ใช้งานอย่างแท้จริง

\section{การประเมินระบบด้วย penalty}
การประเมินผลระบบจัดตารางสอบที่จะพัฒนาขึ้นมานั้นจะพิจารณาปัจจัยด้านความสมดุลและความเหมาะสมของตารางสอบที่ได้ โดยสามารถกำหนดค่าอันไม่พึงประสงค์ (penalty) ของตารางสอบแต่ละแบบที่เป็นผลลัพธ์จากระบบ
\enskip ตารางสอบที่มีความเหมาะสมมาก ควรจะมี penalty น้อย
\enskip นอกจากจะมีการคำนวณ penalty เพื่อเปรียบเทียบตารางสอบแบบต่างๆ ที่ได้จากระบบแล้ว การคำนวณในลักษณะเดียวกันนี้จะใช้กับตารางสอบดั้งเดิมดังที่ได้กำหนดโดยสำนักทะเบียนและประมวลผลอีกด้วย เพื่อยืนยืนว่าตารางสอบที่ได้จากระบบนั้นมีคุณภาพดีกว่าตารางสอบที่มีอยู่เดิม

\subsection{การคำนวณ penalty}
Penalty ของตารางสอบแต่ละแบบที่ได้จากระบบนั้น คำนวณได้จาก penalty ของตารางสอบสำหรับนักศึกษารายบุคคล ว่ามีความเหมาะสมกับนักศึกษารายนั้นๆ มากน้อยเพียงใด \enskip 
การคิดค่า penalty ของนักศึกษาแต่ละคนนั้น จะใช้ตัวชี้วัดทั้งสิ้น 7 แบบ ซึ่งมีน้ำหนักแตกต่างกันไปตามความไม่พึงประสงค์ที่สรุปได้จากผลสำรวจ เรียงตามน้ำหนักจากมากไปน้อยได้ดังนี้
\begin{enumerate}
    \item จำนวนครั้งที่มีนักศึกษาคนใด ๆ ถูกกำหนดให้สอบสองวิชาในเวลาเดียวกัน
    \item จำนวนวันที่นักศึกษาถูกกำหนดให้สอบเวลา 8.00--11.00\,น. และ 12.00--15.00\,น. ในวันเดียวกัน
    \item จำนวนวันที่นักศึกษาถูกกำหนดให้สอบเวลา 12.00--15.00\,น. และ 15.30--18.30\,น. ในวันเดียวกัน
    \item จำนวนวันที่นักศึกษาถูกกำหนดให้สอบเวลา 8.00--11.00\,น. และ 15.30--18.30\,น. ในวันเดียวกัน
    \item จำนวนครั้งที่นักศึกษาถูกกำหนดให้สอบเวลา 15.30--18.30\,น. และ 08.00--11.00\,น. ในวันรุ่งขึ้น
    \item จำนวนครั้งที่วันเว้นว่างระหว่างการสอบสองวิชาที่ติดกันมีมากกว่า 3 วัน
    \item จํานวนครั้งที่จำนวนนักศึกษาที่สอบในช่วงเวลาใด ๆ มากกว่าจํานวนที่นั่งสอบที่ทางมหาวิทยาลัยสามารถจัดให้ได้\CIreply{หรือจะนับจำนวนที่นั่งที่เกินมาไปเลย?}
\end{enumerate}

\section{การประเมินระบบโดยสอบถามความพึงพอใจของนักศึกษา}
ตัวชี้วัดที่ใช้ในการคำนวณ penalty ของตารางสอบที่ได้จากระบบนั้น อาจจะไม่ครอบคลุมความพึงประสงค์ทุกรูปแบบจากนักศึกษา หรือน้ำหนักของตัวชี้วัดที่ใช้ในการคำนวณ penalty นั้นอาจจะเป็นไปได้หลายรูปแบบ
\enskip การสอบถามความพึงพอใจของนักศึกษานั้นจะช่วยในการปรับปรุงอัลกอริทึมเพื่อให้สามารถเลือกใช้น้ำหนักของตัวชี้วัดที่เหมาะสมมากขึ้นในการจัดตารางสอบ
อีกทั้งยังช่วยยืนยันว่าตัวชี้วัดที่โปรแกรมใช้คํานวณค่าความเหมาะสมของตารางสอบนั้นสอดคล้องกับความต้องการของผู้ใช้งาน 
\enskip โดยการประเมินระบบด้วยวิธีการนี้นั้นจะมีการแสดงตารางสอบหลายแบบเพื่อให้นักศึกษาได้เปรียบเทียบและ
ให้คะแนนความพึงพอใจกับตารางสอบแบบต่าง ๆ ซึ่งตารางสอบที่ใช้ในแบบสอบถามนั้นจะเป็นตารางสอบของนักศึกษาคนที่ตอบแบบสอบถาม 
โดยเป็นตารางสอบที่มีค่า penalty ต่ำที่สุดที่ได้จากระบบ โดยจะมีตารางสอบหลายแบบที่มีค่าน้ำหนักของตัวชี้วัดที่แตกต่างกันออกไป 
และรวมไปถึงตารางสอบดั้งเดิมของสำนักทะเบียนและประมวลผลด้วย
