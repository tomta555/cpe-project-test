\chapter{\ifproject%
\ifcpe การทดลองและผลลัพธ์\else Experimentation and Results\fi
\else%
\ifcpe การประเมินระบบ\else System Evaluation\fi
\fi}

การประเมินระบบสำหรับโครงงานนี้จะเน้นไปที่การประเมินความเหมาะสมของตารางสอบที่ได้จากระบบ
\linebreak
เพื่อเปรียบเทียบกับตารางสอบที่ออกโดยสำนักทะเบียน 
โดยเป้าหมายหลักของตารางสอบที่จัดโดยระบบจะเน้นไปที่วันเวลาสอบของนักศึกษาแต่ละคนตามวัตถุประสงค์ที่ได้กล่าวไว้ในตอนที่ \ref{sec:Objectives}

\section{การประเมินระบบด้วย Penalty}
ในการประเมินผลระบบจัดตารางสอบที่จะพัฒนาขึ้นมานั้นจะมีการประเมินความสมดุลและความเหมาะสมของตารางสอบที่เป็นผลลัพธ์ของระบบ จากค่าเฉลี่ยของ Penalty 
โดยหากตารางสอบมีค่า Penalty มากจะหมายถึงตารางสอบนั้นมีความเหมาะสมน้อย โดยค่า Penalty นั้นจะคำนวณมาจากตารางสอบของนักศึกษาแต่ละคน
จากนั้นจึงนำค่า Penalty ของนักศึกษาแต่ละคนมาเฉลี่ยกัน 
โดยจะมีการนำค่าเฉลี่ยของ Penalty มาเปรียบเทียบกันระหว่างตารางสอบทั้ง 2 แบบ
ได้แก่
\begin{itemize}
    \item ตารางสอบที่เป็นผลลัพธ์จากระบบ
    \item ตารางสอบดั้งเดิมจากสำนักทะเบียน
\end{itemize}

\subsection{การคิด Penalty}
ในการคิดค่า Penalty จากตารางสอบของนักศึกษาคนใด ๆ นั้น จะแบ่งตัวแปรย่อยออกเป็น 5 ตัวแปร 
ซึ่งตัวแปรที่มีความสำคัญมากจะมีค่า Pentlty มาก โดยเรียงตามลำดับความสำคัญจากมากไปน้อยตามผลจากการสำรวจ ดังนี้
\begin{enumerate}
    \item จำนวนครั้งที่มีวันเวลาสอบ 8.00-11.00 และ 12.00-15.00 ในวันเดียวกัน
    \item จำนวนครั้งที่มีวันเวลาสอบ 12.00-15.00 และ 15.30-18.30 ในวันเดียวกัน
    \item จำนวนครั้งที่มีวันเวลาสอบ 8.00-11.00 และ 15.30-18.30 ในวันเดียวกัน
    \item จำนวนครั้งที่มีวันเวลาสอบ 15.30-18.30 และวิชาถัดไปสอบเวลา 08.00-11.00 ในวันรุ่งขึ้น
    \item จำนวนครั้งที่มีวันเว้นว่างระหว่างสอบวิชาถัดไปมากกว่า 3 วัน
\end{enumerate}

\section{การประเมินระบบโดยสอบถามความพึงพอใจของนักศึกษา}
ในการประเมินผลระบบจัดตารางสอบที่จะพัฒนาขึ้นมานั้นจะมีการสอบถามความพึงพอใจ
และความคิดเห็นของนักศึกษาต่อตารางสอบจากระบบและตารางสอบดั้งเดิมจากสำนักทะเบียนเพื่อเปรียบเทียบความพึงพอใจของตารางสอบทั้งสองแบบโดยจะนำข้อมูลมาสรุปผล เพื่อนำไปเป็นข้อมูลอ้างอิงสำหรับช่วย
พัฒนาโปรแกรมต่อไป