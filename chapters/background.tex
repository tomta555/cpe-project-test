\chapter{\ifcpe ทฤษฎีที่เกี่ยวข้อง\else Background Knowledge and Theory\fi}
การทำโครงงานนี้เริ่มต้นจากการที่เราเล็งเห็นปัญหาของตารางสอบปลายภาคมหาวิทยาลัยเชียงใหม่ในปัจจุบันซึ่งอาจจะสามารถแปลงเป็นปัญหาที่มีวิธีแก้ไขมาแล้วได้ เราจึงเริ่มต้นศึกษาค้นคว้าทฤษฎีที่เกี่ยวข้องที่สามารถแก้ไขปัญหานี้ได้หรือ งานวิจัย โครงงาน ที่เคยมีผู้นำเสนอไว้แล้ว
ซึ่งเนื้อหาในบทนี้ก็จะเกี่ยวกับการอธิบายถึงสิ่งที่เกี่ยวข้องกับโครงงานนี้้เพื่อให้ผู้อ่านเข้าใจเนื้อหาในบทถัด ๆ ไปได้ง่ายยิ่งขึ้น
\section{Literature review}
การกำหนดเวลาสอบปลายภาคเพื่อหลีกเลี่ยงปัญหานักศึกษาคนใด ๆ มีเวลาสอบในช่วงเวลาเดียวกันสามารถแปลงปัญหานี้ให้เป็นปัญหา graph coloring~\cite{mcs} ได้ 
โดยที่จุดแต่ละจุดในกราฟเป็นรายวิชาที่เปิดสอนในภาคการศึกษานั้น 
และเส้นที่เชื่อมแต่ละจุดสองจุดในกราฟแสดงถึงการมีนักศึกษาที่ลงทะเบียนเรียนทั้งสองรายวิชา โดยจุดสองจุดใด ๆ ในกราฟที่มีเส้นเชื่อมกันจะถูกกำหนดสีให้ต่างกัน
ซึ่งแสดงถึงวันและเวลาที่จัดสอบปลายภาคในวิชานั้น ๆ โดยจุดที่มีคนละสีก็จะถูกจัดให้สอบคนละช่วงเวลากัน

การแปลงปัญหาการจัดตารางสอบปลายภาคให้เป็นปัญหา graph coloring จะสามารถแก้ปัญหาการจัดตารางสอบแบบพื้นฐานได้เท่านั้น โดยไม่คำนึงถึงข้อจำกัดอย่างอื่น 
ตัวอย่างเช่น ไม่พิจารณาความจุที่นั่งสำหรับการสอบแต่ละช่วงเวลาของนักศึกษา จำนวนอาจารย์ที่คุมสอบแต่ละช่วงเวลา และการกระจายวิชาสอบสำหรับนักศึกษาแต่ละคน เป็นต้น
ถึงแม้จะไม่กำหนดข้อจำกัดใด ๆ ปัญหา graph coloring ก็เป็นปัญหา NP-complete ด้วยตัวมันเองอยู่แล้ว~\cite{alg-design} 
ซึ่งหมายความว่ายังไม่สามารถหาอัลกอริทึมที่ใช้เวลา polynomial-time ในการแก้ไขปัญหาให้ได้ผลลัพธ์ที่เหมาะสมที่สุด 
ทำให้ต้องใช้วิธีอื่นที่ให้ผลลัพธ์ที่ดีในระดับที่ยอมรับได้ แต่สามารถยืนยันได้ว่าจะได้วิธีการที่สามารถแก้ไขปัญหาได้อย่างแน่นอน 
ซึ่งวิธีการนั้นคือ metaheuristic ซึ่งสามารถหาวิธีการแก้ปัญหาที่ดีได้ในระยะเวลาที่เหมาะสม~\cite{meta-for-vertexcolor}
และสามารถกำหนดข้อจำกัดหรือเงื่อนไขอื่นเพิ่มเติมได้ ทำให้สามารถกำหนดขอบเขตของผลลัพธ์ได้ แต่อาจจะไม่ได้วิธีแก้ปัญหาที่ดีที่สุด

อีกวิธีการที่สามารถใช้แก้ไขปัญหาการจัดตารางสอบได้ก็คือการใช้ memetic algorithms (MA) ซึ่งเป็นวิธีการที่นำ local search มาประยุกต์ใช้กับ genetic algorithm 
เพื่อช่วยลดระยะเวลาให้คำตอบของปัญหานั้น converge ช้าลง \cite{pablo-memetic-algo} ซึ่ง Ender {\"O}zcan เคยได้นำวิธีการนี้มาประยุกต์ใช้ในการแก้ปัญหาการจัดตารางสอบ 
โดยได้สร้าง Framework สำหรับออกแบบตัวดำเนินการที่ใช้ในการ crossover และ mutation ของ genetic algorithm ด้วย~\cite{fes}
โดยในการทดลองนี้ได้มีการคำนึงถึงนักศึกษา โดยกำหนดข้อจำกัดของตารางสอบที่ได้ให้ไม่มีนักศึกษาที่ต้องสอบติดกันสองวิชาในแต่ละวัน แต่การจัดตารางสอบในแบบของ Ender 
นั้นเป็นการจัดตารางสอบของมหาวิทยาลัย Yeditepe โดยแบ่งตารางสอบเป็นของภาควิชาต่าง ๆ และแบ่งย่อยแยกตามสาขาวิชาอีกที วิธีการนี้ไม่สามารถนำมาใช้กับการจัดตารางสอบของมหาวิทยาลัยเชียงใหม่ได้
เนื่องจากมหาวิทยาลัยเชียงใหม่ มีวิชาศึกษาทั่วไปซึ่งเป็นวิชาที่เปิดให้นักศึกษาจากต่างคณะสามารถลงทะเบียนได้ทำให้มีนักศึกษาเป็นจำนวนมากกว่าที่ความจุที่นั่งสอบของคณะนั้นจะรับไหว
ซึ่งทำให้การจัดตารางสอบโดยใช้วิธีนี้นั้นเป็นไปได้ยากเพราะจำนวนนักศึกษาที่เกินความจุที่นั่งสอบนั้นจะละเมิดข้อจำกัดที่กำหนดไว้ 

\iffalse
\section{Tools}
\subsection{Gurobi Optimizer}
Gurobi Optimizer เป็น Solver ที่ใช้สำหรับแก้ปัญหา optimization โดยที่จะเน้นไปทางด้านของปัญหาต่าง ๆ ดังนี้ 
\begin{itemize}
  \item Linear programming (LP)
  \item Mixed-integer linear programming (MILP)
  \item Quadratic programming (QP)
  \item Mixed-integer quadratic programming (MIQP)
  \item Quadratically-constrained programming (QCP)
  \item Mixed-integer quadratically-constrained programming (MIQCP)
\end{itemize}
ผลลัพธ์ที่ได้จาก Gurobi Optimizer อาจนำมาใช้เป็นตัวเปรียบเทียบประสิทธิภาพกับผลลัพธ์การทำงานที่ได้จากอัลกอลิทึมของเรา
\fi
\section{หลักการ แนวคิด และทฤษฏีที่เกี่ยวข้องในการทำโครงงาน }
\subsection{Metaheuristic}
Meta­heuristic \cite{metaheuris} เรียกได้ว่าเป็นหลักการการแก้ไขปัญหาโดยทั่วไปและยังสามารถหาผลลัพธ์ของ optimization problems ได้อย่างเหมาะสม
หลักการแก้ไขปัญหาเหล่านี้นั้นจะเป็นการเรียนแบบหลักการของธรรมชาติและนำมาแปลงเป็นอัลกอริทึม อย่างเช่น genetic algorithm, evolutionary computation, simulated annealing, tabu search เป็นต้น
\subsection{Genetic algorithms}
Genetic algorithm \cite{ga} เป็นหนึ่งใน metaheuristic ซึ่งเป็นเทคนิคสำหรับค้นหาผลลัพธ์หรือคำตอบโดยประมาณของปัญหา โดยอาศัยหลักการจากทฤษฎีวิวัฒนาการจากชีววิทยาและหลักการคัดเลือกตามธรรมชาตินั่นคือ 
สิ่งมีชีวิตที่เหมาะสมที่สุดจึงจะอยู่รอด กระบวนการคัดเลือกได้เปลี่ยนแปลงสิ่งมีชีวิตให้เหมาะสมยิ่งขึ้นด้วยตัวปฏิบัติการทางพันธุกรรม เช่น การสืบพันธุ์ , การแลกเปลี่ยนยีน , การกลายพันธุ์  


โดยขั้นตอนการทำงานของ genetic algorithm มีดังนี้ 
\begin{enumerate}
  \item initial population เป็นขั้นตอนเริ่มต้นอัลกอริทึมซึ่งจะทำการกำหนดชุดข้อมูลผลลัพธ์ที่ควรจะได้จากการแก้ไขปัญหาซึ่งเรียกชุดข้อมูลนี้ว่า ชุดข้อมูลที่เหมาะสม หรือ chromosome ชุดข้อมูลนี้อาจจะสุ่มขึ้นมาโดยที่ให้ข้อมูลแต่ละ bit เป็นข้อมูล 0 หรือ 1 เรียกข้อมูล bit นี้ว่า genes และนำ genes หลายแท่งมาต่อกันเป็นสาย chromosome 
  \item fitness function จะเป็น function สำหรับการคัดเลือกชุดข้อมูลที่เหมาะให้สามารถอยู่ต่อไปได้ซึ่งจะใช้ fitness scores เป็นเกณฑ์ในการคัดเลือกข้อมูลที่เหมาะสม
  โดย fitness scores จะขึ้นอยู่กับความพอใจในผลลัพธ์ที่ได้ของผู้พัฒนา
  \item genetic operator คือวิธีการในการปรับเปลี่ยนรูปแบบโครงสร้างของชุดข้อมูลที่เหมาะสมสำหรับรุ่นถัดไปซึ่งก็มีวิธีการทำอยู่ 3 แบบหลัก ๆ 
  \begin{itemize}
  \item selection เป็นการเลือกข้อมูลที่เหมาะสมแล้วปล่อยข้อมูลนั้นส่งต่อข้อมูลไปยังรุ่นถัดไป โดยจะเลือกข้อมูลจาก fitness scores
  \item crossover เป็นการสุ่มเลือกข้อมูลแต่ละ bit จากชุดข้อมูลสองตัวเพื่อส่งต่อไปยังรุ่นต่อไป
  \item mutation เป็นการสุ่มข้อมูลบาง bit ของชุดข้อมูลให้มีค่าของข้อมูลตรงข้ามกลับค่าเดิม
\end{itemize}
\end{enumerate}
อัลกอริทึม genetic algorithm สามารถจบการทำงานได้หลายวิธี โดยมีเงื่อนไขจบการทำงานดังนี้
\begin{itemize}
  \item จบการทำงานเมื่อชุดข้อมูลที่เหมาะสมถึงรุ่นที่ต้องการ 
  \item จบการทำงานเมื่อชุดข้อมูลที่เหมาะสมไม่มีการพัฒนาแล้วหรือไม่มีการเปลี่ยนแปลงไปในทางที่ดีขึ้นเป็นเวลานาน
  \item จบการทำงานเมื่อ fitness scores ของชุดข้อมูลที่เหมาะสมถึงค่าที่ต้องการ
\end{itemize}
\subsection{Graph coloring problem}
Graph coloring problem เป็นปัญหาที่เกี่ยวกับการพยายามระบายสีบนจุดของกราฟ โดยให้จุดที่อยู่ติดกันมีสีต่างกันและใช้สีให้น้อยที่สุด
การระบายสีกราฟอาจมีหลายรูปแบบ บางรูปสามารถใช้สีเพียงสองสีก็เพียงพอที่จะให้จุดที่อยู่ติดกันมีสีต่างกัน บางรูปจำเป็นต้องใช้หลายสีถึงจะเพียงพอที่จะให้จุดที่อยู่ติดกันมีสีต่างกัน 
ดังนั้นจะเรียกจำนวนสีอย่างน้อยที่สุดที่เพียงพอที่จะให้จุดที่อยู่ติดกันมีสีต่างกันว่า จำนวนสีของกราฟ
\section{\ifcpe%
ความรู้ตามหลักสูตรซึ่งถูกนำมาใช้หรือบูรณาการในโครงงาน
\else%
ISNE knowledge used, applied, or integrated in this project
\fi
}
\begin{itemize}
  \item 261218 algorithm for computer engineering ได้นำวิธีดำเนินงาน, หลักการและทฤษฏี ดังนี้มาใช้เพื่อแก้ไขปัญหาในโครงงานนี้  
  \begin{itemize}
  \item วิธีการคิดและกระบวนวิเคราะห์ปัญหา
  \item หลักการในการแก้ไขปัญหา
  \item วิธีการแปลงปัญหาใหญ่ให้กลายเป็นปัญหาที่เล็กกว่าเพื่อง่ายต่อการแก้ไขปัญหา
  \item graph coloring
  \item NP-complete
  \end{itemize}

\item 261218 discrete mathematics for computer engineering  ได้นำทฤษฏี ดังนี้มาใช้เพื่อแก้ไขปัญหาในโครงงานนี้
  \begin{itemize}
  \item graph coloring
  \end{itemize}
\end{itemize}

\section{\ifcpe%
ความรู้นอกหลักสูตรซึ่งถูกนำมาใช้หรือบูรณาการในโครงงาน
\else%
Extracurricular knowledge used, applied, or integrated in this project
\fi
}
ความรู้นอกหลักสูตรที่ใช้สำหรับการแก้ไขปัญหาของโครงงานเพื่อให้ได้ผลลัพธ์ที่เหมาะสุดที่สุด เราได้ทำการศึกษา หลักการ ทฤษฏี ดังนี้
\begin{itemize}
  \item meta­heuristic 
  \item genetic algorithm
  \item local search
  \item memetic algorithm
\end{itemize}
